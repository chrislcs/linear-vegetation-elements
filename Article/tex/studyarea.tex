\section{Data and study area}
\subsection{LiDAR and orthophoto data}
Raw LiDAR point cloud data were retrieved from ``Publieke Dienstverlening op de Kaart'' (PDOK), an open geo-information service of the Dutch government.\footnote{https://www.pdok.nl/nl/ahn3-downloads} The data are part of the ``Actueel Hoogtebestand Nederland 3'' (AHN3) dataset, which is collected between 2014 and 2019. The density of the LiDAR data is around 10 pulses/m$^2$ and includes multiple discrete return values (which can result into effective point densities of over 20 point/m$^2$) as well as intensity data. The dataset is collected in the first quarter of each year when deciduous vegetation is leafless \citep{AHN2016inwinjaren}. Nevertheless, the return signal is sufficiently strong to retrieve a useful scan of the vegetation cover. Freely available very high resolution (VHR) true color orthophotos from PDOK with a resolution of 25cm were consulted for validation purposes.\footnote{https://www.pdok.nl/nl/service/wms-luchtfoto-beeldmateriaal-pdok-25-cm-rgb}

All data were analyzed using free and open source software.\footnote{https://github.com/clucas111/delineating-linear-elements} The scripting was performed in Python (3.6.5) using the NumPy (1.14.2) \citep{walt2011numpy}, SciPy (1.1.0) \citep{jones2001scipy}, pandas (0.22.0) \citep{mckinney2010data}, scikit-learn (0.19.1) \citep{pedregosa2011scikit}, and CGAL (4.12) \citep{cgal2018} libraries. PDAL (1.7.2) \citep{pdal2018} was used for preprocessing and downsampling data. CloudCompare (v2.10alpha) \citep{cloudcompare2018} was used for visualizing the point cloud and for the manual classification.

\begin{figure}
	\centering
	\includegraphics[width=\columnwidth]{./img/studyarea}
	\caption[]{Location of the study area in the central part of The Netherlands. The true color aerial photo (PDOK) in the center shows several linear objects in the rural landscape related to agricultural fields, grasslands, bare soil and infrastructure such as (un)paved roads and farmhouses. The numbered photos show a selection of the variety of linear vegetation elements, such as
	\begin{enumerate*}[(1)]
		\item green lanes, 
		\item planted high tree lines along ditches, 
		\item low and high shrubs/copse,
		\item hedges and
		\item rows of fruit trees and willows which are slightly separated from each other.
	\end{enumerate*}}
	\label{fig:studyarea}
\end{figure}

\subsection{Study area}
The case study area is located in a rural landscape in the center of the Netherlands (figure \ref{fig:studyarea}). The area is about 1.6 km from east to west and 1.2 km from north to south, spanning an area of almost 2 million square meters. The point density of the point cloud in the area is 22.49 points/m$^2$. The area contains numerous linear vegetation elements of varying geometry, ranging from completely straight to curved, isolated or connected to other linear or nonlinear objects. Examples of vegetation and non-vegetation elements are planted forest patches, hedges, green lanes, isolated farms, ditches, a river, dykes and a road network (figure \ref{fig:studyarea}). This heterogeneity within a small area ensured that both the classification of vegetation and delineation of linear objects can be efficiently trained and tested.