\section{Concluding remarks}
At present, LiDAR datasets still differ in quality, content and accessibility across and within countries. Therefore, object identification methods developed should overcome these inconsistencies. The quality of the AHN3 dataset of the Netherlands is sufficient to correctly identify linear vegetation objects with our method. In addition, multi-temporal LiDAR datasets can effectively be analyzed for change in the spatial distribution of linear vegetation objects using such a generic classifier. Initiatives to upscale the classification of linear vegetation objects, reed beds and selected forest metrics to national and European scale, based on classification of LiDAR point clouds, and using efficient cloud computing facilities are being made \citep{kissling2017eecolidar}.

The ecological value of providing such large a dataset of linear vegetation objects lies in the broad extent and fine-scale locational details, which is a powerful quality that can be used in the (3D) characterization of ecosystem structure. Existing ecosystem and biodiversity assessment projects, such as the MAES (Mapping and Assessment of Ecosystems and their Services) project \citep{maes2013mapping}, the SEBI (Streamlining European Biodiversity Indicators) project \citep{biala2012streamlining}, and the high nature value farmland assessment \citep{paracchini2008high} on a European scale and assessments of Planbureau voor de Leefomgeving (PBL) on a national level \citep{bouwma2014biodiversiteit}, could profit from the new details.