\section{Discussion}
\subsection{Vegetation classification}
Trimming the data based on the scatter feature proved an efficient step to reduce the computation time needed to classify the vegetation. A substantial part of the point cloud (about 22\%) corresponding mainly to smooth and planar areas such as bare soil, grassland and water bodies, was removed. This preprocessing step made the dataset imbalanced, but proper steps could be taken to handle the problems when classifying such datasets. When analyzing the classification statistics it is important to take this filtering step into consideration. The removed points are the ones easy to classify as not belonging to tall vegetation. Consequently, the remaining points that do not belong to tall vegetation (\textit{other}) share many similarities with the tall vegetation points and are therefore harder to classify correctly. The filtered points are not part of the accuracy assessment and therefore these statistics might give a distorted, more negative picture of the accuracy. If the filtered points would be part of the assessment it would show even higher accuracy values, but it would give less information on how the classification actually went.

Nevertheless, the majority of points were correctly classified (table \ref{tab:confmatclass}). Most of the incorrectly classified points are dispersed and surrounded by correctly classified points. These scattered points did not have a major impact on the subsequent segmentation of linear objects.

A limitation in our accuracy assessment is the lack of a validation dataset that was taken from a completely different area. Such a validation dataset is an effective measure against overfitting. To prevent overfitting we instead used a cross validation which, while considered an effective approach, is often less preferred than a validation dataset. (need ref)

\subsection{Linear object segmentation}
The comparison of the manual with automated delineation shows that linear objects were accurately extracted (figure \ref{fig:segmentation}), which is supported by the accuracy scores. The largest errors are found at places where a nonlinear element transitions into a linear one. This is a consequence of the rectangularity sometimes only falling below the threshold after a while of growing in a wrong direction.

\medskip
(needs more work)
\medskip

It is important to keep in mind that the definition of a linear element is not clearly defined and is therefore somewhat arbitrary. The accuracy assessment was based on our interpretation of the linear elements in the area. 

\medskip
(needs more work)
\medskip

% However, the linear objects at 1 and 2 (figure ..) show profound omissions when compared to the manual classification. Closer inspection of the rectangular objects in this area (figure ..) clarifies that a series of rectangular, but disconnected objects have been segmented instead of a single object. These objects are by themselves nonlinear and were not merged, because they are not aligned. This omission is attributed to the local pattern of the planted trees in rows of two to three across a local dike. Evidently, the objects are not always merged correctly. The merging is done based on the orientation and alignment of two adjacent and elongated objects, which works well in most cases, but not in this case. Separated single trees are also a problem, because a single tree has no meaningful orientation. Consequently a row of separated single trees will not be merged and subsequently not be classified as a linear vegetation object. The omissions at 3, 4 and 5 are the result of the downsampling of the points, since this can cause small objects to be ignored. The forest patches at 7 and 8 were automatically classified as linear elements, because of side-effects along the boundary of the study area, but would otherwise be classified correctly.